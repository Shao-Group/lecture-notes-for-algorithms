\setcounter{definition}{0} \setcounter{property}{0} \setcounter{claim}{0} \setcounter{fact}{0} \setcounter{corollary}{0} \setcounter{figure}{0}
\section{Knapsack Problem}

\subsection*{The Problem}

Given a knapsack with total capacity of $W$, and $n$ items, where item-$i$ has weight $w_i$ and value $v_i$,
the \emph{knapsack problem} is to select a bunch of items, represented as a vector $(x_1, x_2, \cdots, x_n)$,
where $x_i$ gives how many item-$i$ are picked, such that
the total weight of these items is at most $W$, i.e., $\sum_{1\le i \le n} x_iw_i \le W$,
and that the total value of them, i.e., $\sum_{1\le i \le n} x_i v_i$, is maximized.

There are two variants of above knapsack problem. 
The first variant~(with repetition) allows each (type of) item can be
selected unlimitted number of times, i.e., $x_i \in \{0, 1, 2, \cdots \}$, $1\le i \le n$.
The second variant~(without repetition) requires that each item can be
selected at most once, i.e., $x_i \in \{0, 1\}$, $1\le i \le n$.

Consider an instance with total capacity $W = 7$ and 4 items with weights being $(5, 2, 1, 3)$
and values being $(8, 3, 2, 7)$. 
For the variant~1, the optimal solution is $(0, 0, 1, 2)$, i.e.,
item~3 is picked once and item~4 is picked twice, which gives a total value of 16 and a total weight of 7.
For the variant~2, the optimal solution is $(0, 1, 1, 2)$, i.e.,
item~2, 3 and 4 are picked (once), which gives a total value of 12 and total weight of 6.

\subsection*{An Exercise}

Since we want to maximize the total value given the capacity limit,
an intuitive strategy is therefore to pick ``the most valuable item per unit''.
This leads to a greedy algorithm. Formally, we sort all items w.r.t.\ $v_i/w_i$ in descending order,
and then pick items (greedily) following this order: if the current item can be fit into the knapsack,
we pick it, reduce the total capacity by its weight, and then either stay at this item~(for variant~1) or move to next item~(for variant~2);
if the weight of the current item exceeds the remaining capacity of the knapsack, move to the next item. 
This algorithm is good heuristic and unfortunately may not always find the optimal solution.
\vspace*{-\topsep}
\begin{enumerate}
\item Design an instance for variant~1 such that this algorithm fails to find the optimal solution.
\item Design an instance for variant~2 such that this algorithm fails to find the optimal solution.
\end{enumerate}

\subsection*{Algorithm for Variant~1}

We now design an optimal, dynamic programming algorithm for variant~1.
Again, let's first check if it satisfies optimal substructure property.
Let $p = (x_1, x_2, \cdots, x_n)$ be the optimal solution.
Assume that $x_k \ge 1$. Let $q$ be the solution obtained from $p$ by
reducing $x_k$ by 1, i.e., $q = (x_1, x_2, \cdots, x_{k-1}, x_k - 1, x_{k + 1}, \cdots, x_n)$.
Then $q$ is the optimal solution when the total capacity is $W - w_k$.
(Proof: suppose $q$ is not the optimal solution with total capacity being $W - w_k$,
i.e., one can find a solution achieves higher value than $q$ and does not exceed capacity limit of $W - w_k$.
Then adding item-$k$ to this solution yields a better solution than $p$, contradicting to the assumption of $p$.)

According to this optimal substructure property, naturally,
we define $F(a)$ as the total achievable value with capacity limit being $a$.
Clearly $F(W)$ answers the original question.
We now develop the recursion.
In order to calculate $F(a)$, following above optimal substructure property,
we need to find an item in the optimal solution so as to reduce it to a smaller subproblem.
Which item is in the optimal solution? We don't know. So we enumerate.
We consider every single item that fits~(i.e., weight is less or equal to $a$).
Once we know, say item-$k$, is in the optimal solution, the remaining part of the optimal
solution can be obtained by calculating the optimal solution with $a - w_k$ being the total capacity.
Hence, the recursion is
$$F(a) = \textstyle \max_{1\le k \le n, w_k \le a} (F(a-w_k) + v_k).$$

The algorithm simply fills the array of size $(W + 1)$, i.e., $F(0), F(1), \cdots F(W)$,
with this recursion. The running time is therefore $O(nW)$, as solving
a single subproblem takes $O(n)$ time.

Note that this algorithm does not run in polynomial-time.
This is because, the \emph{input size} of numerical value $W$ is $\log W$.
By rewriting the running time as a function of input-size, i.e.,
$O(nW) = O(n2^{\log W})$, we can see the running time is exponential.
(Note: if the running time is a polynomial function of \emph{numerail values}, we call it
\emph{pseudo-polynomial time} algorithm. And this algorithm is such an example.)

\subsection*{Algorithm for Variant~2}

Note that the optimal substructure property stated above for variant~1 does not hold for variant 2 anymore.
This is because in variant 2 each item can be picked at most once. Therefore, another parameter should
be introduced to ``memorize'' which items are available in the subproblem. A convenient and efficient way is to
limit the largest index of available items. Here is the optimal substructure property for variant 2:
let $p = (x_1, x_2, \cdots, x_n)$ be the optimal solution~(recall that each $x_i \in \{0,1\}$).
Let $k$ be the largest index with $x_k = 1$.
Let $q$ be the solution obtained from $p$ by
reducing $x_k$ by 1, i.e., $q = (x_1, x_2, \cdots, x_{k-1}, x_k - 1, x_{k + 1}, \cdots, x_n)$.
Then $q$ is the optimal solution when the total capacity is $W - w_k$ and only first $(k-1)$ items are available.
(Proof: suppose $q$ is not the optimal solution,
i.e., one can find a set of items from the first $(k-1)$ items that achieves higher value~(than $q$) and does not exceed capacity limit of $W - w_k$.
Then adding item-$k$ to this solution yields a better and feasible solution than $p$, contradicting to the assumption of $p$.)

Following above optimal substructure property,
we define $F(k, a)$ as the total achievable value using the first $k$ items with capacity limit being $a$.
Clearly $F(n, W)$ answers the original question.
We now develop the recursion.
Consider the last step, i.e., whether item-$k$ is included in the optimal solution.
In the case that the optimal solution doesn't include item-$k$ we have $F(k,a) = F(k-1,a)$.
In the case that the optimal solution includes item-$k$ we have $F(k,a) = F(k-1,a-w_k) + v_k$.
Note that in order to be able to include item-$k$, it is required that $w_k \le a$.
Hence, the recursion is
\begin{displaymath}
F(k,a) = 
\left\{
	\begin{array}{lllll}
	 \max \left\{
			\begin{array}{lllll}
			F(k-1,a-w_k) + v_k\\
			F(k-1,a)
			\end{array}
		\right. & \textrm{ if } w_k \le a \\
	F(k-1,a) & \textrm{ if } w_k > a
	\end{array}
\right.
\end{displaymath}

The algorithm fills out the DP table of size $n(W + 1)$
with above recursion. The running time is therefore $O(nW)$, as solving
a single subproblem takes $O(1)$ time.
Again this is a \emph{pseudo-polynomial time} algorithm. 


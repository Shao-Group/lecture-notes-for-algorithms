\documentclass[letterpaper,11pt]{article}

\usepackage{import}

\usepackage{geometry}
\usepackage{pslatex}
\usepackage{graphicx}
\usepackage{color}
\usepackage{tikz}
\usepackage{setspace}
\usepackage{amssymb}

\geometry{ margin = 1.0in }

%\pagestyle{fancy}
%\lhead{{\bf Lecture 41}}
%\chead{{\bf CMPSC 465 Fall 2020}}
%\rhead{{\bf Mingfu Shao}}

\usepackage{fancyhdr}
%\setlength{\headheight}{15pt}

\pagestyle{fancy}
%\renewcommand{\chaptermark}[1]{ \markboth{#1}{} }
\renewcommand{\sectionmark}[1]{ \markright{#1} }

\fancyhf{}
\fancyhead[LE,RO]{\thepage}
\fancyhead[RE]{{ \nouppercase{\leftmark}} }
\fancyhead[LO]{{ \nouppercase{\thesection:~\rightmark}} }

\renewcommand{\thesection}{Lecture~\arabic{section}}

%\fancypagestyle{plain}{ %
%  \fancyhf{} % remove everything
%  \renewcommand{\headrulewidth}{0pt} % remove lines as well
%  \renewcommand{\footrulewidth}{0pt}

\fancypagestyle{titlestyle}
{
	\fancyhf{}
	\renewcommand{\headrulewidth}{0pt}
}

\fancypagestyle{tablestyle}
{
	\fancyhf{}
	\fancyhead[LE,RO]{\thepage}
	\fancyhead[RE]{{ \nouppercase{\leftmark}} }
	\fancyhead[LO]{{ \nouppercase{\rightmark}} }

   %\fancyhf{}
   %\lhead{{\bf Lecture Notes for Algorithms}}
   %\rhead{{\bf Mingfu Shao}}
}


\setlength\parindent{0em}
\setlength\parskip{8pt}
%\setlength{\fboxsep}{6pt}

\usepackage{amsthm}
\newtheoremstyle{mytheorem}
  {\parskip} % Space above
  {0em} % Space below
  {} % Body font
  {} % Indent amount
  {\bfseries} % Theorem head font
  {.} % Punctuation after theorem head
  {.5em} % Space after theorem head
  {} % Theorem head spec (can be left empty, meaning `normal')

\theoremstyle{mytheorem}
\newtheorem{definition}{Definition}
\newtheorem{property}{Property}
\newtheorem{claim}{Claim}
\newtheorem{fact}{Fact}
\newtheorem{corollary}{Corollary}

% for algorithms
\newcommand{\aaa}[1]{\hspace{0.65cm}\parbox[t]{15.3cm}{#1}}
\newcommand{\aab}[1]{\hspace{1.15cm}\parbox[t]{15.0cm}{#1}}
\newcommand{\aac}[1]{\hspace{1.65cm}\parbox[t]{15.0cm}{#1}}
\newcommand{\aad}[1]{\hspace{2.15cm}\parbox[t]{15.0cm}{#1}}
\newcommand{\aae}[1]{\hspace{2.65cm}\parbox[t]{15.0cm}{#1}}
\newcommand{\aaf}[1]{\hspace{3.15cm}\parbox[t]{15.0cm}{#1}}
\newcommand{\aaA}[2]{\hspace{0.5cm} {\tikz[overlay] \draw (0.1, -0.1) -- (0.1, #1 * -1.5em + 0.6em);} \parbox[t]{15.0cm}{#2}}
\newcommand{\aaB}[2]{\hspace{1.0cm} {\tikz[overlay] \draw (0.1, -0.1) -- (0.1, #1 * -1.5em + 0.6em);} \parbox[t]{15.0cm}{#2}}
\newcommand{\aaC}[2]{\hspace{1.5cm} {\tikz[overlay] \draw (0.1, -0.1) -- (0.1, #1 * -1.5em + 0.6em);} \parbox[t]{15.0cm}{#2}}
\newcommand{\aaD}[2]{\hspace{2.0cm} {\tikz[overlay] \draw (0.1, -0.1) -- (0.1, #1 * -1.5em + 0.6em);} \parbox[t]{15.0cm}{#2}}
\newcommand{\aaE}[2]{\hspace{2.5cm} {\tikz[overlay] \draw (0.1, -0.1) -- (0.1, #1 * -1.5em + 0.6em);} \parbox[t]{15.0cm}{#2}}
\newcommand{\xxx}{\par\vspace{0.1cm}}

\begin{document}

% title page

\thispagestyle{titlestyle}
\begin{center}
\vspace*{4cm} {\huge \bf Lecture Notes for Algorithms } \vspace*{2cm}

{\large  Mingfu Shao}

{\large  The Pennsylvania State University}

{\large  \today}
\end{center}

\clearpage \newpage


\makeatletter
\renewcommand{\l@section}{\@dottedtocline{1}{0em}{5em}}
\makeatother

\thispagestyle{tablestyle}
\tableofcontents \thispagestyle{tablestyle} \clearpage \newpage

\addcontentsline{toc}{section}{\textbf{Introduction}}
\import{Lecture01/}{content.tex} \clearpage\newpage
\import{Lecture02/}{content.tex} \clearpage\newpage

\addcontentsline{toc}{section}{\textbf{Divide-and-Conquer}}
\import{Lecture03/}{content.tex} \clearpage\newpage
\import{Lecture04/}{content.tex} \clearpage\newpage
\import{Lecture05/}{content.tex} \clearpage\newpage
\import{Lecture06/}{content.tex} \clearpage\newpage

\addcontentsline{toc}{section}{\textbf{Connectivity of Graphs}}
\import{Lecture07/}{content.tex} \clearpage\newpage
\import{Lecture08/}{content.tex} \clearpage\newpage
\import{Lecture09/}{content.tex} \clearpage\newpage
\import{Lecture10/}{content.tex} \clearpage\newpage

\addcontentsline{toc}{section}{\textbf{Shortest Path Problems}}
\import{Lecture11/}{content.tex} \clearpage\newpage
\import{Lecture12/}{content.tex} \clearpage\newpage
\import{Lecture13/}{content.tex} \clearpage\newpage
\import{Lecture14/}{content.tex} \clearpage\newpage
\import{Lecture15/}{content.tex} \clearpage\newpage
\import{Lecture16/}{content.tex} \clearpage\newpage

\addcontentsline{toc}{section}{\textbf{Dynamic Programming}}
\import{Lecture17/}{content.tex} \clearpage\newpage
\import{Lecture18/}{content.tex} \clearpage\newpage
\import{Lecture19/}{content.tex} \clearpage\newpage
\import{Lecture20/}{content.tex} \clearpage\newpage
\import{Lecture21/}{content.tex} \clearpage\newpage
\import{Lecture22/}{content.tex} \clearpage\newpage

\addcontentsline{toc}{section}{\textbf{Greedy Algorithms}}
\import{Lecture23/}{content.tex} \clearpage\newpage
\import{Lecture24/}{content.tex} \clearpage\newpage
\import{Lecture25/}{content.tex} \clearpage\newpage
\import{Lecture26/}{content.tex} \clearpage\newpage
\import{Lecture27/}{content.tex} \clearpage\newpage
\import{Lecture28/}{content.tex} \clearpage\newpage

\addcontentsline{toc}{section}{\textbf{Network Flow}}
\import{Lecture29/}{content.tex} \clearpage\newpage
\import{Lecture30/}{content.tex} \clearpage\newpage
\import{Lecture31/}{content.tex} \clearpage\newpage

\addcontentsline{toc}{section}{\textbf{Computational Complexity}}
\import{Lecture32/}{content.tex} \clearpage\newpage
\import{Lecture33/}{content.tex} \clearpage\newpage


\end{document}
